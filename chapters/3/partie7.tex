\section{Challenges and Solutions}

\subsection{Access Restrictions}
Initially, access to the development environment was limited due to company security policies. This issue was resolved by migrating the project to a local environment and collecting necessary resources, such as screenshots and documentation.

\subsection{Data Synchronization}
Ensuring real-time data synchronization between controllers and the platform was a challenge due to network latency. This was addressed by optimizing MQTT communication and implementing efficient data caching strategies.

\section{Conclusion}
The development of the LAB.E.S platform involved the integration of various technologies, from modern web development tools to industrial automation systems. By adopting a microservices architecture and utilizing robust frameworks like .NET and ReactJS, the platform successfully achieved its goal of providing a scalable and efficient solution for laboratory management.

