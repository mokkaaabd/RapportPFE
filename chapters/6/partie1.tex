

\section{Poste, Tâches et Responsabilités}
I assumed the position of developer for my LAB.E.S. project, which involved real-time supervision and the implementation and optimization of the test bench infrastructure. Creating automated methods to guarantee dependable testing and effective data collecting procedures was my primary responsibility. Understanding user needs, creating the system architecture, and using contemporary components like the iBox and CP343-1 Lean to enhance communication between various infrastructure components were among my duties.
As a technical team member, I actively participated in a number of departmental tasks. In my free time, I also offered suggestions to the design team to enhance the real-time monitoring platform's user interface. In order to guarantee dependable connections, I also had to supervise the administration of network communications between devices.
Working with the management department to enhance the distribution of laboratory performance and findings was another facet of my involvement.


\section{Formation}
In order to completely comprehend the constraints and difficulties of the current infrastructure, my training started with an examination of the settings and protocols that were already in place. After that, I looked at research and case studies to determine which best practices to use for my project. I gained a thorough understanding of real-time supervisory techniques during this time, as well as the significance of technology decisions for performance improvement.
In a second stage, I attended seminars in the working environment to become acquainted with various industrial devices and received comprehensive training on how to use the iBox. This helped me improve my ability to adjust to real-world work settings and improve my technical abilities for a smooth project integration.

\section{Résultats et Analyse}
Throughout my internship, I encountered a number of difficulties. The biggest obstacle was adjusting to new integrated technologies, including the CP343-1 Lean module, which necessitated plc programming expertise.
Meeting the project's strict budgetary criteria was a further significant challenge. Cost limits and the acquisition of contemporary technology were frequently incompatible, necessitating a rigorous evaluation of priorities and decisions.
Here is a list of the primary difficulties:

1.Adaptability to changes: Testing procedures are slowed down because it frequently takes time to modify test bench setups.
2. Complexity of equipment communication: In the case of unexpected failures or breakdowns, controlling linked equipment became challenging.
3. Lack of specialized training: Not all team members had the specialized training needed to use some outdated equipment, which caused delays in the process.

