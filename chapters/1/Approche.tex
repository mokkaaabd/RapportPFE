\section{Aproche de travail}
After introducing the company and defining the scope of the project, it is critical to describe the working strategy used to bring this project to completion. We picked the Scrum technique, which is an agile approach to complicated project management designed to promote flexibility, cooperation, and efficiency. This chapter teaches Scrum concepts, team responsibilities, and how to plan and monitor project progress using a Gantt chart.

\subsection{Principe de Scrum}
Scrum is an agile project management methodology that uses iterative and incremental approaches. The primary characteristics of Scrum are:

\begin{enumerate}
  \item Sprints: Work iterations that extend from two to four weeks in which we build and deliver an element of the product.
  \item Increment: At the end of each sprint, a functional increment that adds value is delivered. 
  \item Regular meetings: In addition to sprint planning meetings, sprint reviews, and retrospectives, I collaborate weekly with Mr Ludwig Taeubert, Full Stack Developer, who offers technical guidance. These meetings allow me to track the status of assignments, fix technological issues and improve coordination.
 
\end{enumerate}




\subsection{Rôles dans une Équipe Scrum}
To manage the project, we used the Scrum agile methodology, which allows for an iterative, incremental approach. This technique is founded on a number of important ideas and responsibilities that have guided our development team.
\begin{description}
  \item[Product Owner (PO):] Mr Khaled Abdelwaheb, Plant Laboratory Manager. He is responsible for defining product requirements and managing the backlog. His role is to ensure that business needs are correctly translated into functionalities.
  \item[Scrum Master:] Mr Maxime Loren, Digital Engineering Manager. He facilitates the Scrum process, supports the team in removing obstacles and ensures the application of agile best practices.
  \item[Development team:] The development team, of which I am the sole member, is in charge of providing functional increments each sprint. We are in charge of technological implementation and job fulfillment in accordance with the sprint backlog.
\end{description}


\subsection{Utilisation du Diagramme de Gantt}
Although Scrum focuses on iterative development and adaptability, we also used a Gantt chart to plan and monitor the project's key stages. The tool gave a detailed picture of projects, deadlines, and dependencies, allowing the team to coordinate more effectively. The Gantt chart for our project was organized as follows:

\begin{enumerate}
  \item Phase Definition: The project was divided into four primary phases: initial planning, development, testing, and release. These phases have been clearly outlined to help advancement.
  \item Tasks and Subtasks: Each phase was divided into distinct tasks and subtasks that were explicitly assigned to members of the development team. This made it easier to assign roles and track individual efforts.
  \item Tracking Deadlines and Progress: The Gantt chart was used to monitor the progress of the individual activities, detect any delays, and alter priorities as needed to ensure that the overall project deadlines were reached.
  \item Dependency Management: Task dependencies, such as the one between requirements collecting and finalization, were clearly specified. This allowed for more proactive risk management and interdependence among project stages.
\end{enumerate}



